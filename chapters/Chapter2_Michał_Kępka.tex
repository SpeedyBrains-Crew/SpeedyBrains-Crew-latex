\section{Michał Kępka}
\label{sec:mkepka}

Cute dog image (see Figure~\ref{fig:dog}).

\begin{figure}[h]
    \centering
    \includegraphics[width=0.4\textwidth]{pictures/dog.jpeg}
    \caption{Cutest dog in the entire world}
    \label{fig:dog}
\end{figure}

Table~\ref{tab:grades}: contains students' grades

\begin{table}[h]
\centering
\begin{tabular}{lllll}
\hline
Person   & Grade 1 & Grade 2 & Grade 3 &  \\ \hline
person 1 & 4       & 3       & 5       &  \\
person 2 & 3       & 2       & 6       &  \\
person 3  & 5       & 4       & 6       &  \\ \hline
\end{tabular}
\label{tab:grades}
\caption{Table representing grades of the students}
\end{table}

\begin{equation}
    \centering
    \int_a^b \! f'(x) \, \mathrm{d}x = f(b) - f(a)
    \label{mat:calculus_equation}
\end{equation}

List 1
\begin{enumerate}
    \item Element 1
    \item Element 2
    \item Element 3
    \item Element 4
\end{enumerate}
\vspace{20pt}

List 2
\begin{itemize}
    \item[:)] Element 1
    \item[:|] Element 2
    \item[:o] Element 3
    \item[:(] Element 4
\end{itemize}

\paragraph{Learn C++.}
We recommend reading this tutorial, in the sequence listed in the left menu.
C++ is an \textbf{object oriented} language and some concepts may be new. Take breaks when needed, and go over the examples as many times as needed.
C++ is a \textbf{cross-platform} language that can be used to create high-performance applications.
C++ was developed by \textbf{Bjarne Stroustrup}, as an extension to the C language.
C++ gives programmers a \textit{high level of control over system resources and memory}.
The language was updated 4 major times in 2011, 2014, 2017, and 2020 to C++11, C++14, C++17, C++20.

\paragraph{C++ Install IDE.}
An IDE (Integrated Development Environment) is used to edit AND compile the code.
Popular IDE's include Code::Blocks, Eclipse, and Visual Studio. These are all free, and they can be used to both edit and debug C++ code.
Note: Web-based IDE's can work as well, but functionality is limited.
We will use Code::Blocks in our tutorial, which we believe is a good place to start.
You can find the latest version of \textit{Codeblocks} at http://www.codeblocks.org/. Download the mingw-setup.exe file, which will install the text editor with a compiler.



